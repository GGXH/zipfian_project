\documentclass[12pt]{article}

\begin{document}

\title{Project Proposal: Will the Foreign Currency Exchange Rate raise or fall tomorrow?}
\author{Xin Guo}
\date{\today}
\maketitle

Prediction of the tendency in financial market is an old and important problem.  A well prediction of the market (e.g.\ foreign currency exchange market or stock market) may lead to large capital.  As the development of computer and machine learning techniques, it is 



%1.  High level description of the project: what question or problem are you addressed:
%
%•	I will build up a model to predict the tendency of the next day exchange rate of USD/EUR, USD/GBP, and USD/YEN.
%•	Which features (e.g., financial index) are important to predict the tendency of the Foreign Currency Exchange Rate?
%
%2. How are you presenting your work?
%
%•	The final product will be a summary of the work flow (what I do to build up the model), the accuracy of my model, and the prediction results of my model.
%•	(Optional)  If the time allowed, a web site can be built to predict the next day tendency by using the live stream data (if no time for this, it will become future work)
%
%3.  What are your data resources?
%
%•	My data can range from 2000-last week.  The data set contain the bid and sell price of each tick.
%
%3. What is your next step?
%
%•	In my analysis, I will only use the highest, lowest, close, and mean prices of each day.  Therefore, the first step is to clean up my data set to extract the four important numbers for each day to reduce the data size.
%•	Use the highest, lowest, close, and mean prices to calculate features of the data set, totally about 100 features will be generated, examples of features are as follows:
%o	Stochastic Oscillator (SO), e.g. a 14 day-period SO = (Current Close – Lowest Low) / (Highest High – Lowest Low) * 100, the highest high is the highest price in the 14 day-period and the lowest low is the lowest price in the 14 day-period.
%o	Momentum = current closing price - oldest closing price 
%o	Local slopes of the price time history based on different time period
%o	Price in previous several days
%•	Use feature selection techniques (e.g., SVM-RFE, random forest) and feature extraction techniques (e.g., SVD) to reduce the feature size to 5 to 20.
%•	Train clustering models (e.g., logistic regression, SVM, BPNN, RBFNN)
%•	Final results will show which combination (feature selection/extraction-clustering models) has the best prediction.
\end{document}